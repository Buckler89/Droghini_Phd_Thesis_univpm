\documentclass[]{letter}


\begin{document}
% If you want headings on subsequent pages,
% remove the ``%'' on the next line:
% \pagestyle{headings}

\begin{letter}{Thesis Abstract}
\opening{"Ambient Intelligence:
Computational Audio Processing
For Human Fall Detection"}

At present, Ambient Intelligence represents the challenge of the future. To obtain an ecosystem that is fully functional and calibrated to the user need, numerous systems, each of them dedicated to a specific task, must be integrated. One of these sub-systems is the human fall detection. Both research community and governments gave particular attention to the human fall detection because the fall is the first cause of death for people over 65. In this thesis, the human fall detection is addressed from an audio perspective: a dataset named A3FALL, composed of a corpus of several audio fall events of every-day objects and both simulated and real human falls recorded in 3 different rooms, has been presented. In particular, a special floor acoustic sensor (FAS) has been developed from this purpose and used to record the dataset together with an array of a microphone array.
Different approaches that work with a different knowledge base according to the specific task have been proposed: first, two supervised approaches have been described that have highlighted the peculiarities of the audio drop detection and demonstrated the effectiveness of the proposed sensor.
The human falls hardly available for systems development, unsupervised systems have been proposed that do not need examples of the target class in the learning phase. It has been shown that unsupervised approaches have better performance than the art state systems, but they do work well in not very complex scenarios.
Finally, methods that work under more realistic conditions have been developed and described. A system where the user intervenes by correcting the system's operation for a considerable reduction of false alarms is proposed. Then a few-shot learning approach that without any user intervention can achieve promising results using only a few examples of human fall in the training phase has been presented. The thesis concludes with an extensive evaluation of a Siamese Convolutional Autoencoder based approach. It has been shown that this approach outperforms all the previously proposed systems when assessed in a complex scenario.


\signature{Fabio Vesperini}

%\closing{}

%enclosure listing
%\encl{}

\end{letter}
\end{document}
