\chapter{Introduction}\label{ch:intro}

Outline, obiettivi, controbuti

The decreasing birth rate \cite{eurostat} and the contemporary increase of the life expectancy at birth \cite{Carone2006} in the majority of industrialized countries have been generating new challenges in the assistance of the elderly. The scientific community, companies and governments are trying to face them by investing in the development of efficient healthcare systems and solutions. The direction taken goes towards the development of smart home capable of taking care of the inhabitants by supporting and monitoring them in their daily actions \cite{Dawadi20161188, Principi2015a}. Since falls are one of the main cause of death for the elderly \cite{mubashir2013survey}, several efforts have been devoted to the development of algorithms for automatically detecting these events.





\section{Fall Detection Systems}
The continuous and unprecedented growth rate of the elderly world population is one of the primary aspects of concern for society and governments. Nowadays about 8.5\% of people in the world are more than 65 years old \cite{dhhsOlderPop,Carone2006}. Although the average life of the world population is getting longer, elderly people may not    necessarily live a healthier life. It is enough to say that 37.5 million falls require medical interventions and more than 600 thousand are cause of death every year worldwide. In particular, the population segment most affected by this problem is composed of elderly over 65 years that, with the growing mobility of the population, are more frequently left alone in their homes without aid in the case of need. Moreover, since falls are the leading cause of death and hospitalizations for older adults, this phenomenon leads to a substantial increase in the cost of healthcare \cite{whoFall, mubashir2013survey}. 
It is not surprising, thus, that the research community is encouraged, even by governments, to find reliable and performing solutions to minimize the damage caused by the human falls problem. This is also confirmed by the presence in the literature of several reviews dedicated to this specific topic \cite{mubashir2013survey, khan2017review, lapierre2017state, pannurat2014automatic, xu2018new, el2013fall}.
In fact, in the past few years, a variety of systems have been presented. One way to divide the methodologies for approaching the falls detection problem is based on the placement of the sensing devices \cite{mubashir2013survey}. The main categories are wearable, vision and environmental, with each category presenting their own advantages and disadvantages. Wearable systems do not suffer from ambient condition, but people may forget to wear them and they are not operational during the charging time, thus, some people may consider them annoying. Furthermore, a device must be installed on each person to be monitored. An environmental sensor may be used to avoid this kind of problems, but with other limitations. Vision systems, although they are actually environmental sensors, deserve a dedicated category because of many systems proposed in the literature based on this type of sensors \cite{mubashir2013survey}. This category includes several types of sensors like, e.g., cameras for which the major limitations are field-of-view constraints, lighting condition, positioning of multiple cameras and lack of privacy.
The ambient category includes several types of sensors. For example, radar doppler based systems used in \cite{wu2015radar} raise fewer privacy concerns, but they suffer from reflection and blind spots. In particular, for a data-driven system, another aspect that should not be underestimated is the need for a re-training when changing the environment to be monitored or even just some of its components such as the arrangement of furniture as happens in \cite{liu2008vision}.
All this implies that there is no optimal choice, which is instead, a compromise that depends on the type of environment that is monitored as well as on personal sensitivity of the subjects under monitoring.
Going into more detail, another significant distinction between falls detection systems can be made based on the type and amount of data used for the algorithm development \cite{khan2017review}. In fact, the problem can be approached either as supervised or unsupervised based on the availability of data in the hands of the researchers as well as their goals. 
Most state-of-the-art works tackle the problem under fully supervised conditions assuming they have enough data for falls. Almost all of these falls are simulated with professional mannequins \cite{werner2011fall, zigel2009method}  or by people with adequate protections \cite{li2012microphone, popescu2008acoustic} that however may not correctly emulate an actual fall. Although this approach leads to more accurate results, there is no guarantee that it will generalize well in real situations. 
Other researchers opt for approaches based on outlier/anomaly detection \cite{khan2015unsupervised, zhang2009detecting, popescu2009acoustic} because of the plentiful availability of data that can represent normal activity. However, it is challenging to define what ``normal activities'' are for such approaches, and the risk is to raise several false alarms. % come succede a noi con la ocsvm.
Perhaps the situation that most closely approximates reality is a hybrid between the previous ones, in which a large amount of data representing the normality are easily available, with just a few samples of real human fall (\textit{RHF}) and eventually some related synthetic or simulated data. In these situations, supervised approaches that suffer from strong data imbalance have to apply subsampling \cite{stone2015fall} or weighting \cite{khan2017review} techniques to mitigate this effect. Thus, the need to find an effective way to exploit the few available falls data is evident.


\section{State-Of-The-Art} 
Review dei sistemi per la fall detection basati sui vari tipi di sensori
accelerometers, vision, ambient. Per gli ambient paricolare enfasi sugli approcci basati su aduio.
\subsection{Problem Statement}


