\documentclass[]{letter}


\begin{document}
	% If you want headings on subsequent pages,
	% remove the ``%'' on the next line:
	% \pagestyle{headings}
	
	\begin{letter}{Thesis Abstract}
		\opening{"Ambient Intelligence:
Computational Audio Processing
For Human Fall Detection"}
		
	L' Ambient Intelligence rappresenta la sfida del futuro. Per ottenere un ecosistema funzionante e calibrato in base alle esigenze dell'utente, è necessario integrare numerosi sistemi, ciascuno dedicato a un compito specifico. Uno di questi sottosistemi è il rilevamento della caduta umana. In questa tesi, il rilevamento delle cadute umane è affrontato da una prospettiva audio. In questo lavoro è stato presentato un set di dati denominato A3FALL, composto da una serie di diversi eventi audio relativi alla caduta di oggetti comuni e cadute umane, simulate e reali. In particolare, per tale scopo è stato sviluppato uno speciale sensore acustico a pavimento (FAS) che è stato utilizzato per registrare il set di dati insieme ad un array di microfoni.
	Sono stati proposti differenti approcci che lavorano con una diversa base di conoscenze in base al compito specifico: in primo luogo sono stati descritti due approcci supervisionati che hanno evidenziato le peculiarità del rilevamento della caduta audio e hanno dimostrato l'efficacia del sensore proposto.
	Per via delle poche cadute umane reali a disposizione per lo sviluppo di tali sistemi, sono stati proposti approcci non supervisionati che non necessitano di esempi della classe target nella fase di apprendimento. È stato dimostrato che gli approcci non supervisionati hanno prestazioni migliori rispetto ai sistemi dello stato dell'arte, ma funzionano bene in scenari poco complessi.
	Infine, sono stati sviluppati e descritti metodi che funzionano in condizioni più realistiche. Viene proposto un sistema in cui l'utente interviene correggendo il funzionamento del sistema per una notevole riduzione dei falsi allarmi. Poi viene proposto un approccio di apprendimento one-shot che, senza l'intervento dell'utente, può ottenere risultati promettenti utilizzando solo alcuni esempi di caduta umana nella fase di addestramento. La tesi si conclude con una valutazione approfondita di un approccio basato su un Autoencoder siamese. È stato dimostrato che questo approccio è migliore di tutti i sistemi precedentemente proposti quando valutatti in  scenari complessi.
		
		
		\signature{Fabio Vesperini}
		
		%\closing{}
		
		%enclosure listing
		%\encl{}
		
	\end{letter}
\end{document}
