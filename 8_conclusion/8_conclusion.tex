 
\chapter{Conclusions}

In this dissertation, the problem of human fall detection has been widely addressed. An innovative sensor named FAS was initially proposed \secref{sec:sensor}. Its operating principle is similar to the one of stethoscopes: a membrane is in contact with the floor, i.e., the transmitting medium of the fall waves and a resonance enclosure accommodates a microphone that captures the waves and converts them in electrical signals. The floor sensor minimizes the impact of aerial sounds into the audio recordings, making it suited to deal with sounds induced by falls. In order to evaluate the effectiveness of the floor acoustic sensor and due to the lack of publicly available audio dataset for fall detection, we have created a dedicated corpus comprises recordings of fall events related to everyday objects and background noises. The sounds have been collected in 3 different rooms with different acoustic characteristics. The human fall has been simulated by means of a human mimicking doll. The dataset has been made publicly available to the research communities. By analyzing the signals collected in the database \secref{ssec:sig_analysis}, we have highlighted the behaviour of the FAS that show a good SNR at low frequencies with respect to the other standard microphones. That allowed to work at low sampling frequencies, considerably reducing the computational cost of the algorithms. In particular, in \chref{ch:supervised_approaches} the two types of microphones were compared using SVM-based supervised approach. It has been shown that by working at lower sampling rates the FAS can achieve better results than the other microphones due to its acoustic properties. However, human falls are tough to find in a real situation. The supervised approaches are therefore not feasible. In chapter x, therefore, methods have been proposed that work in the opposite condition: the non-supervised approaches indeed can work without examples of the target class for their training. In \secref{sec:ocsvm_approach} an OCSVM method has been proposed and trained with background noises only. Although the proposed system can work well when the test set is composed by only the same categories used for the training plus the human fall class, performance decays when the test set is composed of signals not included in the normality model, such as events of falling objects. Indeed, the challenge in such approaches is to have enough data to shape normality. In \secref{sec:autoencoder} has been proposed a different novelty detection approach that works in an end-to-end learning process. A different novelty detection approach that works with an end-to-end learning process. This system employs a neural network autoencoder and to remove the need for handcrafted features. It has been shown that this system outperforms the OCSVM based method in the less difficult scenario where the teste is composed only of the background noises and the human fall signals.