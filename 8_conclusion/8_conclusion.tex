 
\chapter{Conclusions}

In this dissertation, the problem of human fall detection has been widely addressed. An innovative sensor named FAS was initially proposed in \secref{sec:sensor}. Its operating principle is similar to the one of stethoscopes: a membrane is in contact with the floor, i.e., the transmitting medium of the fall waves and a resonance enclosure accommodates a microphone that captures the waves and converts them in electrical signals. The floor sensor minimizes the impact of aerial sounds into the audio recordings, making it suited to deal with sounds induced by falls. In order to evaluate the effectiveness of the floor acoustic sensor and due to the lack of publicly available audio dataset for fall detection, we have created a dedicated corpus comprises recordings of fall events related to everyday objects and background noises. The sounds have been collected in 3 different rooms with different acoustic characteristics. The human fall has been simulated by means of a human mimicking doll and also performed by 4 volunteers for a total of 124 (44 simulated and 80 real). The dataset has been made publicly available to the research communities. By analyzing the signals collected in the database \secref{ssec:sig_analysis}, we have highlighted the behaviour of the FAS that show a better SNR at low frequencies with respect to the other standard microphones. That allowed to work at low sampling frequencies, considerably reducing the computational cost of the algorithms. In particular, in \chref{ch:supervised_approaches} the two types of microphones were compared by using SVM-based supervised approach. It has been shown that by working at lower sampling rates the FAS can achieve better results than the other microphones due to its acoustic properties. However, human falls are tough to find in a real situation; the supervised approaches are therefore not feasible. Hence, in \chref{ch:unsupervised_approaches} methods that work in the opposite condition have been proposed: the non-supervised approaches indeed can work without examples of the target class for their training. In \secref{sec:ocsvm_approach} an OCSVM method has been proposed and trained with background noises only. Although the proposed system can work well when the test set is composed by only the same categories used for the training plus the human fall class, performance decays when the test set is composed of signals not included in the normality model, such as events of falling objects. Indeed, the challenge in such approaches is to have enough data to shape normality. In \secref{sec:autoencoder} has been proposed a different novelty detection approach that works in an end-to-end learning process. This system employs a neural network autoencoder to remove the need for handcrafted features. It has been shown that this system outperforms the OCSVM based method in the less difficult scenario where the test is composed only of the background noises and the human fall signals. 
In \chref{ch:weakley_supervised} have been proposed the systems that work in an intermediate condition of knowledge compared to previous ones. In \secref{sec:user_aided_cin} has been proposed an extension of the OCSVM based system: a user-aided approach for template matching able to reduce the false alarm rate with respect to the base method. Template matching stage requires user feedback on several false alarm prediction. The system memorizes signaled audio segments to use them in future classifications in order to improve performance reducing the true negatives.
These preliminary results prompted us to further investigate the Siamese approach. In \secref{sec:siamese_one_shot} an improvement is proposed that provide a convolutional Siamese autoencoder (SCAE) for metric learning.  The whole A3FALL dataset was used (chapter x)  to assess the system. In particular,  the system is able to exploit the similes between simulated and real falls, making a useful transformation for the training of the downstream Knn classifier. In this way, only a real fall is required in the training set. In this very complex scenario, has been shown that supervised approaches are not able to guarantee reliable performance as well as unsupervised approaches. Contrariwise basic Siamese approach confirmed to be robust, while its proposed SCAE improvement gave a further boost to performance.
In conclusion, in this thesis, a path of innovation was presented regarding approaches dedicated to human fall detection, which followed a  data availability perspective. We started from facilitated conditions in which all samples were labeled and available for each class to end with a system that faces one of the major problems regarding the fall classification task: the scarce availability of real human falls. Then we got to the development of a system ready to be evaluated in a real environment.  The proposed sensor can be embedded on the bottom of a ground lamp and made invisible to the user's eye. The algorithm can be integrated with an interoperability framework to become an integral part of an Ambient Intelligence system. Although the audio-based approaches can work stand-alone, in an AmI vision, these approaches can support other fall detection systems that are based on different sensors such as vision or wearable, to obtain a very robust and reliable system.